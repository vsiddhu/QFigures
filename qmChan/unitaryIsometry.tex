\documentclass{article}

\usepackage{tikz}
\usetikzlibrary{arrows,shapes,positioning}

\usepackage{verbatim}
\usepackage[tightpage,active]{preview}
\PreviewEnvironment{tikzpicture}
\setlength\PreviewBorder{.5mm}

\newcommand{\HC}{{\mathcal H}}
\newcommand{\ket}[1]{|#1\rangle }
\newcommand{\ot}{\otimes }

%Title          :   Unitary generating an isometry
%Author         :   Vikesh Siddhu
%Email          :   vsiddhu@andrew.cmu.edu
%Description    :   Unitary U from bipartite input a-e to b-c spaces. Fixing
%                   the e input of U gives an isometry J from a to b-c.


\begin{document}
\pagestyle{empty}

\tikzstyle{int}=[circle,draw, fill=white, minimum size=.1em]
\tikzstyle{Jbox} = [draw=black, fill=blue!20, minimum height=.8cm, minimum width=.8cm]

\begin{tikzpicture}[node distance=0.5cm,auto,>=latex']
% Figure (a): Unitary
    \node [int] at (0,1) (a) {$\HC_a$};
    \node [above of = a, node distance=.6cm] (aT) {\footnotesize in};
    \node [] at (0,0) (e) {$\ket{e_0}$};
    \node [Jbox] at (1.3,1) (b) {$U$};
    \node [int] at (2.5,2) (c) {$\HC_b$};
    \node [below of = c, node distance=.6cm] (bT) {\footnotesize out};
    \node [int] at (2.5,0) (d) {$\HC_c$};
    \node [above of = d, node distance=.6cm] (cT) {\footnotesize env.};
    
    \path[->] (a) edge              node         {} (b);
    \path[->] (e) edge              node         {} (b);
    \path[->] (b) edge              node         {} (c);
    \path[->] (b) edge              node         {} (d);
    
    \node[align = left, font=\footnotesize] at (1.2, -1.2) (cap_a) 
    {Unitary $U:\HC_a \ot \HC_e \mapsto \HC_b \ot \HC_c$ \\ defines an isometry
    $J:\HC_a \mapsto \HC_b \ot \HC_c$ \\ with $J\ket{\psi} = U \ket{\psi} \ot
    \ket{e_0}$};
    
    \end{tikzpicture}
\end{document}
