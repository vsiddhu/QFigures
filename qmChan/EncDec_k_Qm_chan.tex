\documentclass{article}
\usepackage{tikz}
\usetikzlibrary{arrows,shapes,positioning}

\newcommand{\AC}{{\mathcal A}}
\newcommand{\DC}{{\mathcal D}}
\newcommand{\EC}{{\mathcal E}}
\newcommand{\HC}{{\mathcal H}}
\newcommand{\NC}{{\mathcal N}}

\usepackage{verbatim}
\usepackage[tightpage,active]{preview}
\PreviewEnvironment{tikzpicture}
\setlength\PreviewBorder{.5mm}

%Title          :   Encoding and Decoding across k uses of a Quantum Channel
%Author         :   Vikesh Siddhu
%Email          :   vsiddhu@andrew.cmu.edu
%Description    :   Quantum states from a Hilbert space H are encoded via E(k)
%                   into the input of k parallel uses of a channel B and 
%                   decoded from the joint output via D(k)

\begin{document}
\pagestyle{empty}

\tikzstyle{int}=[circle,draw, minimum size=3.25em]
\tikzstyle{chan1}=[rectangle,draw,  minimum width=1.3cm, minimum height=6cm]
\tikzstyle{chan2}=[rectangle,draw,  minimum width=.8cm, minimum height=.8cm]
\tikzstyle{txBox} = [draw=none, fill=none, minimum height=1em, minimum width=1em]

\begin{tikzpicture}[node distance=.75cm,auto,>=latex']
    \node [int] at (0,0) (a) {$\HC^{(k)}$};
    \node [below of = a] (aT) {Input};

    \node [chan1] at (2,0) (b) {$\EC^{(k)}$};

    \node [txBox] at (3.2,0) (c) {$k$};
    \node [chan2] at (4,2.5) (c1) {$\NC$};
    \node [chan2] at (4,1.5) (c3) {$\NC$};
    \node [chan2] at (4,-2.5) (c2) {$\NC$};
    
    \node [chan1] at (6,0) (d) {$\DC^{(k)}$};

    \node [int] at (8,0) (e) {$\HC^{(k)}$};
    \node [below of = e] (aT) {Output};
    
    \path[->] (a) edge              node         {} (b);
    \draw[->, dotted, thick] (c) -- (3.2,2.5) ;
    \draw[->, dotted, thick] (c) -- (3.2,-2.5) ;
    
    \draw[->] (2.65,2.5) -- (c1) ;
    \draw[->] (2.65,1.5) -- (c3) ;
    \draw[->] (2.65,-2.5) -- (c2) ;
    \draw[->] (c1) -- (5.35,2.5) ;
    \draw[->] (c3) -- (5.35,1.5) ;
    \draw[->] (c2) -- (5.35,-2.5) ;

    \path[dotted, thick] (c3) edge              node  {} (c2);
    \path[->] (d) edge              node         {} (e);

    \end{tikzpicture}
\end{document}
