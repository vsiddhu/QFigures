\documentclass{article}

\usepackage{tikz}
\usetikzlibrary{arrows, positioning}
\newcommand{\sg}{\sigma }
\newcommand{\ket}[1]{|#1\rangle }
\newcommand{\mte}[2]{\langle#1|#2|#1\rangle }
\newcommand{\BC}{{\mathcal B}}
\newcommand{\CC}{{\mathcal C}}

\usepackage{amsmath}
\usepackage{verbatim}
\usepackage[tightpage,active]{preview}
\PreviewEnvironment{tikzpicture}
\setlength\PreviewBorder{.5mm}
 
%Title          :   Qubit dephasing channel and its complement
%Author         :   Vikesh Siddhu
%Email          :   vsiddhu@andrew.cmu.edu
%Description    :   For a qubit input with Bloch vector r_a = (x,y,z), the
%                   dephasing channel has output Bloch vector 
%                   r_b = (x, (1-2p)y, (1-2p)z), and complementary output vector
%                   r_c = (2z sqrt(p(1-p)), 0, 1-2p)
%                   where p = 0.3
%                   The input Bloch sphere with origin O. Unit length Bloch vectors
%                   along +x and −x axis represent [-] and [+] respectively. In
%                   this notation [psi] is |psi><psi|, |+> = (|0> + |1>)/sqrt(2)
%                   |-> = (|0> - |1>)/sqrt(2)
\begin{document}
\pagestyle{empty}

\begin{tikzpicture}[scale=2]

%X Axis
\path (0.1,0) edge[line width=.1mm, ->] node[minimum size=.1mm, fill=white, anchor=center, pos=0.75, font=\normalsize] {$+x$}(1,0);    
\path (-0.1,0) edge[line width=.1mm, ->] node[minimum size=.1mm, fill=white, anchor=center, pos=0.75, font=\normalsize] {$-x$}(-1,0);    

% A input Sphere
\coordinate (O) at (0, 0) {};
\draw [dashed, line width=.05mm] (O) ellipse(.25 and 1);
\draw[line width=.05mm] (O) circle(1);
%
\node[] at (0,0) (Or) {O};
\node[] at (0,1.2) (Otk) {\normalsize $[0]$};
\node[] at (0,-1.2) (Obk) {\normalsize $[1]$};
\node[] at (1.2,0) (Otk) {\normalsize $[+]$};
\node[] at (-1.2,0) (Obk) {\normalsize $[-]$};
    
% B output 
\draw [line width=.1mm, red] (O) ellipse(1 and .4);
\draw [line width=.1mm, red, dashed] (O) ellipse(.1 and .4);

%B output label

% C output
\coordinate (Oc1) at (.92, .405) {};
\coordinate (Oc2) at (-.92, .405) {};
\path[line width=.1mm, blue] (Oc1) edge (Oc2);    

%C output label
\node[] at (1.1, .405) (Otk) {\normalsize $[\chi_0]$};
\node[] at (-1.1, .405) (Obk) {\normalsize $[\chi_1]$};

%
\node[red] at (2.5, 0.8) (BC) {\normalsize $\BC(\rho) = (1-p)\rho + p \sg_x \rho \sg_x$};
\node[red] at (2.08,0.4) (sg) {\normalsize $\sg_x = \left( \begin{matrix} 0&1\\1&0
\end{matrix} \right)$ };
\node[blue] at (2.85,-.5) (CC) {\normalsize $\CC(\rho) = \mte{+}{\rho}\, [\chi_+] + \mte{-}{\rho}\, [\chi_-]$};
\node[blue] at (2.52,-.8) (chi) {\normalsize $\ket{\chi_{\pm}} = \sqrt{1-p} \; \ket{0} \pm \sqrt{p} \; \ket{1}$};

\node[align=left] at (1,-1.75) () {\normalsize Caption: Enclosed inside a Bloch sphere
    (in black), an ellipsoid (in red) and line (in blue)\\ representing the locus
    of Bloch vectors of $\BC$ and $\CC$ respectively where $p = 0.3$};

\end{tikzpicture}

\end{document}


