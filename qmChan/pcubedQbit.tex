\documentclass{article}

\usepackage{tikz}
\usetikzlibrary{arrows, positioning}

\usepackage{verbatim}
\usepackage[tightpage,active]{preview}
\PreviewEnvironment{tikzpicture}
\setlength\PreviewBorder{.5mm}

\newcommand{\BC}{{\mathcal B}}
\newcommand{\CC}{{\mathcal C}}
\newcommand{\DC}{{\mathcal D}}
\newcommand{\al}{\alpha }
\newcommand{\bt}{\beta }
\newcommand{\gm}{\gamma }
\newcommand{\ket}[1]{|#1\rangle }

%Title          :   Qubit pcubed channel
%Author         :   Vikesh Siddhu
%Email          :   vsiddhu@andrew.cmu.edu
%Description    :   A qubit qubit channel taking a qubit input to a qubit output
%                   and a qubit environment. Channel is of the pcubed type, i.e.,
%                   it takes two non-orthogonal pure states at the input to
%                   pure states at the channel output and environment. See
%                   Phys. Rev. A 94, 052331 (2016) for details.

\begin{document}
\pagestyle{empty}

\begin{tikzpicture}[scale=1]
    %INPUT SPHERE
    \coordinate (O) at (0, 0) {};
    \draw [dashed, thin] (O) ellipse(1 and .5);
    \draw[thin] (O) circle(1);
    % Projections onto sphere
    \coordinate (Ps) at (0,1) {};
    % Sphere poles
    \coordinate (N) at (.85,-.5) {};
    % Chord
    \draw[] (N)--(Ps);
    \draw[dashed] (N)--(O);
    \draw[dashed] (O)--(Ps);
%        % labels placed last
    \begin{scope}[inner sep=2pt]
    \node [] at (Ps) [above] {\footnotesize $\ket{\al_1}$};
    \node [right of =N, node distance=.5cm] (xa) {\footnotesize $\ket{\al_2}$};
    \end{scope}
    % dot marks

    %B OUTPUT SPHERE
    \coordinate (Ob) at (7.5, 0) {};
    \draw [black, dashed, thin] (Ob) ellipse(1 and .5);
    \draw[black,thin] (Ob) circle(1);
    % Projections onto sphere
    \coordinate (Psb) at (7.5,1) {};
    % Sphere poles
    \coordinate (Nb) at (8.2, .2) {};
    % Chord
    \draw[black] (Nb)--(Psb);
    \draw[black, dashed] (Nb)--(Ob);
    \draw[black, dashed] (Ob)--(Psb);
%        % labels placed last
    \begin{scope}[inner sep=2pt]
    \node [black] at (Psb) [above] {\footnotesize $\ket{\bt_1}$};
    \node [black] at (Nb) [below] {\footnotesize $\ket{\bt_2}$};
    \end{scope}
    % dot marks

    %C OUTPUT SPHERE
    \coordinate (Oc) at (3.75, -1) {};
    \draw [black, dashed, thin] (Oc) ellipse(1 and .5);
    \draw[black,thin] (Oc) circle(1);
    % Projections onto sphere
    \coordinate (Psc) at (3.75,0) {};
    % Sphere poles
    \coordinate (Nc) at (4.35, -.25) {};
%        % Chord
    \draw[black] (Nc)--(Psc);
    \draw[black,dashed] (Nc)--(Oc);
    \draw[black,dashed] (Oc)--(Psc);
%        % labels placed last
    \begin{scope}[inner sep=2pt]
    \node [black] at (Psc) [above] {\footnotesize $\ket{\gm_1}$};
    \node [right of =Nc, node distance=.5cm, black] (x) {\footnotesize $\ket{\gm_2}$};
    \end{scope}
    % dot marks
    
    \path[->, black] (1,.5) edge [bend left] node[above, black]  {$\BC$} (6.25,.5);
    \path[->, black] (1,-.75) edge [bend right] node[below, black] {$\CC$} (3,-1.75);
    \path[->, red] (6.25,-.5) edge [bend left] node[below right, red] {$\DC$} (4.5,-1.75);

\node[align=left] at (3.75,-3.) () {\footnotesize Caption: Schematic
representation of a degradable qubit channel $\BC$ \\ 
\footnotesize with a qubit complement $\CC$ and degrading map $\DC$.\\ 
\footnotesize For details, see Sec. IV.A in
Phys. Rev. A 94, 052331 (2016)};

\end{tikzpicture}

\end{document}
